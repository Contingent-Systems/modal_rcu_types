\makeatletter % allow us to mention @-commands
\def\arcr{\@arraycr}
\makeatother
\section{Evaluation}
\label{sec:eval}
We have used our type system to check correct use of RCU primitives in two RCU data structures representative of the broader space.

Figure \ref{fig:rculist} gives the type-annotated code for \lstinline|add| and \lstinline|remove| operations on a linked list implementation of a bag data structure, following McKenney's example~\cite{McKenney2015SomeEO}.
Appendix \ref{appendix:bag_paul} contains the code for membership checking.

We have also type checked the most challenging part of an RCU binary search tree, the deletion (which also contains the code for a lookup).
Our implementation is a slightly simplified version of the Citrus BST~\cite{Arbel:2014:CUR:2611462.2611471}: their code supports fine-grained locking for multiple writers, while ours uses only single global lock for writer threads.
For lack of space the annotated code is only in Appendix \ref{appendix:bst_del}, but it motivates some of the conditional-related flexibility discussed in Section \ref{subsection:type-action} \iso{Colin, it would be great if you could articulate the importance of BST delete proof here}.

To the best of our knowledge, we are the first to check such code for memory-safe use of RCU primitives modularly, without appeal to the specific implementation of RCU primitives.
