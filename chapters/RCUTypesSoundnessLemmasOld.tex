%!TEX root = ./paper.tex
\section{Soundness}
\label{sec:lemmas}

\begin{Lemma}[Safe ReadBlock Traversal]
\begin{gather}
\llbracket x:=x.f \rrbracket \lfloor \llbracket \Gamma,x:\mathsf{rcuIterator} \rrbracket_{\textsf{R},tid} * \{m\}\rfloor  \subseteq \\ 
\lfloor \llbracket \Gamma,x:\mathsf{rcuIterator} \rrbracket_{\textsf{R},tid}  * \mathcal{R}(\{m\})\rfloor 
\end{gather}
\end{Lemma}
\begin{proof}
We assume 
\begin{gather} \label{l1e1}
(\sigma,O,U,T) \, \in \, \llbracket \Gamma,x:\textsf{rcuIterator} \rrbracket_{\textsf{R},tid} * \{m\}
\\
\label{l1ewfa}
\textsf{WellFormed}(\sigma,O,U,T)
\end{gather}
And must show that there exists $O',U',T'$ such that
\begin{gather} \label{l1e3}
%\begin{gather}
(\sigma[ \sigma.s(x,tid) \mapsto \sigma.h(\sigma.s(x,tid),f) ], O',U',T') \in \\
\llbracket \Gamma,x:\textsf{rcuIterator} \rrbracket_{\textsf{R},tid} * \mathcal{R}(\{m\})
%\end{gather}
\\
\label{l1ewfa}
\textsf{WellFormed}(\sigma[ \sigma.s(x,tid) \mapsto \sigma.h(\sigma.s(x,tid),f)  ],O',U',T')
\end{gather}
%
From \ref{l1e1} we can assume
%
%%% gather formats a collection of numbered equations nicely
\begin{gather} \label{l1e4}
(\sigma,O_{1},U_{1},T_{1}) \in \llbracket \Gamma, x:\textsf{rcuIterator} \rrbracket_{\textsf{R},tid}
\\
 \label{l1e5}
(\sigma,O_{2},U_{2},T_{2}) = m 
\\
 \label{l1e6}
O_{1} \bullet O_{2} = O
\\
 \label{l1e7}
U_{1} \cup U_{2} = U 
\\
 \label{l1e8}
T_{1} \uplus T_{2} = T
\\
\label{l1wfa1}
\textsf{WellFormed}(\sigma,O_{1},U_{1},T_{1})
\\
\label{l1wfa2}
\textsf{WellFormed}(\sigma,O_{2},U_{2},T_{2})
\end{gather}
%
To prove \ref{l1e3} choose 
%
\[
\begin{array}{cl}
O' = O[ \sigma.s(x,tid) \mapsto O \setminus \textsf{iterator} \, tid  \cup \textsf{iterator} \, tid ] \\
U' = U \\
T' = T
\end{array}
\]
We must prove there exists $O'_{1}, O'_{2}, U'_{1}, U'_{2}, T'_{1}, T'_{2}$ such that
\begin{gather}\label{l1e9}
(\sigma',O'_{1},U'_{1}, T'_{1}) \in \llbracket \Gamma,x:\textsf{rcuIterator} \rrbracket_{R,tid}
\\
\label{l1e10}
(\sigma',O'_{2},U'_{2}, T'_{2}) \in \mathcal{R}(\{m\})
\\
 \label{l1e11}
O'_{1} \bullet O'_{2} = O'
\\
\label{l1e12}
U'_{1} \cup U'_{2} = U' 
\\
\label{l1e13}
T'_{1} \uplus T'_{2} = T
\\
\label{l1wfp1}
\textsf{WellFormed}(\sigma',O_{1}',U_{1}',T_{1}')
\\
\label{l1wfp2}
\textsf{WellFormed}(\sigma',O_{2}',U_{2}',T_{2}')
\end{gather}
%
We choose $O_{2}' = O_{2}$, $U_{1}' = U_{1}$, $U_{2}' = U_{2}$, $T_{1}' = T_{1}$ and $T_{2}' = T_{2}$.
So \ref{l1e12} follows from \ref{l1e7} and \ref{l1e13} follows from \ref{l1e8}.

In the rest of the proof it is useful to know that 
\begin{equation} \label{l1e14}
T_{1} = \{tid\}
\end{equation}
which follows directly from \ref{l1e4}.

To show \ref{l1e11}, pick
\[
O_{1}'  = O_{1}[\sigma.s(x,tid) \mapsto O_{1} \setminus \textsf{iterator} \;  tid \cup \textsf{iterator} \; tid]
\]
$O'_{1} \bullet O'_{2} = O'$, follows, as we know $O'_{2}$ does not include $tid$.

From \ref{l1e4} we can assume 

\begin{gather} \label{l1e18}
(\sigma,\hat{O}'_{1}, U, T) \in \llbracket \Gamma \rrbracket_{R,tid}
\\
 \label{l1e19}
(\sigma,\hat{O}''_{1}, U, T) \in \llbracket x:\textsf{rcuIterator} \rrbracket_{R,tid}
\\
 \label{l1e20}
\hat{O}'_{1} \bullet \hat{O}''_{1} = O_{1}
\end{gather}

To show \ref{l1e9}, we need to show that there exists $O''_{1}$ and $O'''_{1}$ such that
\begin{gather} \label{l1e15}
(\sigma',O''_{1}, U, T) \in \llbracket \Gamma \rrbracket_{R,tid}
\\
 \label{l1e16}
(\sigma',O'''_{1}, U, T) \in \llbracket x:\textsf{rcuIterator} \rrbracket_{R,tid}
\\
 \label{l1e17}
O''_{1} \bullet O'''_{1} = O'_{1}
\end{gather}

Choosing $O''_{1} = \hat{O}'_{1}$, \ref{l1e15} follows  from \ref{l1e18}.

Assume that \[ o = \sigma.s(x,tid)\]

From semantics of \[ \llbracket x:\mathsf{rcuIterator} \rrbracket_{tid} \] we know that \[ \mathsf{iterator} \, tid \in O(o) \land x\notin U \]

By choosing $O'''_{1}$ to contain the new iterator from $O'_{1}$, , we know that

\[ x \notin dom(\Gamma) \]

as \[ \forall y \ldotp \sigma(y) = \sigma'(y) where x \neq y \]

as a result  

\ref{l1e15} implies \ref{l1e9}.

To show \ref{l1e10} we need to show that
\[
(\sigma, O_{2}, U_{2}, T_{2}) \mathcal{R} (\sigma', O_{2}, U_{2}, T_{2})
\]
%$\sigma$ agrees with $\sigma'$ on the heap and the stack for threads other than $tid$. By \ref{l1e8} and \ref{l1e14}, $tid \notin T_{2}$ so this is allowed.
which by definition means we must show
\begin{gather}
\label{l1re1}
\sigma.l \in T_{2} \rightarrow \sigma.h = \sigma'.h
\\
\label{l1re2}
\textsf{iterator} \, tid \in O_{2}(o) \rightarrow  o \in dom(\sigma'.h)
\\
\label{l1re3}
O_{2} = O_{2}
\\
\label{l1re4}
U_{2} = U_{2}
\\
\label{l1re5}
T_{2} = T_{2}
\\
\label{l1re6}
t \in T_{2} \rightarrow \forall x \ldotp \sigma.s(x,t) = \sigma'.s(x,t)
\end{gather}

\ref{l1re3}-\ref{l1re5} hold trivially. To show \ref{l1re1} we know $\sigma' = \sigma [  \sigma.s(x,tid) \mapsto \sigma.h(\sigma.s(x,tid),f)  ]$. So $\sigma'.h = \sigma.h$ as only $\sigma.s$ changes. To show \ref{l1re2} we know by \textsf{\textbf{MustBeAllocated}} that $o \in dom(\sigma.h)$ and thus $o \in dom(\sigma'.h)$. To show \ref{l1re6} we know $T_{1} = \{tid\} \land tid \notin T_{2}$.  We can assume $t \in T_{2} \land t \ne tid$ and must prove that 
\[
\forall x \ldotp \sigma.s(x,t) = \sigma'.s(x,t) 
\]
Which holds by the definition of $\sigma'$ as only $tid$s variables could be updated. 

%%%%%%Equation 21

To show \ref{l1wfp1} we assume \ref{l1wfa1} and pick
\[
O_{1}'  = O_{1}[\sigma.s(x,tid) \mapsto O_{1} \setminus \mathsf{iterator} \,  tid \cup \mathsf{iterator} \, tid]
\]

In addition, we know \ref{l1e14} and $tid  \in R$.

Assume that \[ o = \sigma'.s(x,tid)\]

From semantics of \[ \llbracket x:\mathsf{rcuIterator} \rrbracket_{R,tid} \] we know that \[ \mathsf{iterator} \, tid \in O(o) \land x\notin U \]

(\ref{l1wfp1} $\ldotp$ \textsf{Iterator I}) holds as $\mathsf{iterator} \, tid \in O'_{1}(o)$.

 
\[
O_{1}'  = O_{1}[\sigma.s(x,tid) \mapsto O_{1} \setminus \mathsf{iterator} \,  tid  \cup \mathsf{iterator} \, tid]
\]

\[ o = \sigma.s(x,tid) \in O_{1}\]

\[ o' = \sigma'.s(x,tid) \in O'_{1}\]


 As we know that $o \notin O'_{1}$ but $o' \in O_{1}'$, we can deduce that a thread $tid \in R$ can only have one pointer to a particular memory location.  This implies that (\ref{l1wfp1} $\ldotp$ \textsf{Iterator II}) holds. 

To show (\ref{l1wfp1} $\ldotp$ \textsf{Iterator III}), we know that for some $o$  where $s(x,tid) = o$
\[ \mathsf{iterator} \, tid\, \in O(o) \land  \exists T \ldotp T(T,o) \in F\]
and we need to show that 
\[tid \in T\].

As we know from semantics of 
\[
(\texttt{delayedFree}(x), s,h,l,R,F) 
\;\;\Downarrow_{\mathrm{tid}}\;\;
(s,h,l,R,F \uplus \{ (R,S(x))\} 
\]
that
\[s(x,tid) \in S(x)\]
then we conclude that 
\[tid \in T\]

In addition to (\ref{l1wfp1} $\ldotp$ \textsf{Iterator II}), we know that heap owns all locations that it owns before so  (\ref{l1wfp1} $\ldotp$ \textsf{Ownership}) holds as

\begin{gather*}
 \forall  o'' \, \ldotp \, h(o,f) = v \, \land \, h(o'',f') = v\, \land \, v \in  \mathsf{OID} \, \land \, \mathsf{ftype}(f) = \mathsf{rcu}, \, \implies \\
 o=o'' \land f=f'' \end{gather*}

holds.

(\ref{l1wfp1} $\ldotp$ \textsf{UnlinkedI}), (\ref{l1wfp1}$\ldotp$ \textsf{UnlinkedII}),(\ref{l1wfp1}$\ldotp$ \textsf{UnlinkedIII}), (\ref{l1wfp1}$\ldotp$ \textsf{FreshI}), (\ref{l1wfp1}$\ldotp$ \textsf{Freeing I}),(\ref{l1wfp1}$\ldotp$ \textsf{Freeing II}) and (\ref{l1wfp1}$\ldotp$ \textsf{MustBeAllocated}) are trivial


%%%%%%%%%%%%%Note:Check Equation_21.MustBeAllocated and others. Equation_22 

To show \ref{l1wfp2} we assume \ref{l1wfa2} and we already know that $O_{2}' = O_{2}$,  $U_{2}' = U_{2}$ and $T_{2}' = T_{2}$  .

 We know $tid' \in T_{2}$. (\ref{l1wfp2}$\ldotp$ \textsf{Ownership}), (\ref{l1wfp2}$\ldotp$ \textsf{IteratorI}), (\ref{l1wfp2}$\ldotp$ \textsf{IteratorII}), (\ref{l1wfp2}$\ldotp$ \textsf{UnlinkedI}), (\ref{l1wfp2}$\ldotp$ \textsf{UnlinkedII}),(\ref{l1wfp2}$\ldotp$ \textsf{UnlinkedIII}), (\ref{l1wfp2}$\ldotp$ \textsf{FreshI}), (\ref{l1wfp2}$\ldotp$ \textsf{Freeing I}), (\ref{l1wfp2}$\ldotp$ \textsf{Freeing II}) and (\ref{l1wfp2}$\ldotp$ \textsf{MustBeAllocated}) hold as only $s$ changes and $h$, $F_{2}$, $O_{2}$ do not change. 

\begin{comment}
For proof of  \ref{l1wfp2}$\ldotp$ \textsf{FreshII}, we can assume that 
\[o = O_{2}(x,t) \land o' = O'_{2}(x,t)\]
and we know that 
\[
\sigma' = \sigma [  \sigma.s(x,t) \mapsto \sigma.h(x,f) ]   \land \sigma.h = \sigma.h' 
\]
As heap does not change and assuming that 

\[ \mathsf{fresh} \in O_{2}'(o') \]

there exists only one action that preserves observation of a thread for a node as \textsf{fresh}
\[
\Gamma,\, x:\textsf{fresh } f\,\_ \vdash_{M} x.f := z \dashv x:\textsf{fresh }f\,z \,,\Gamma
\]
 which is a mutator action and this implies that 

\[t = l\]

For another case of \ref{l1wfp2}$\ldotp$ \textsf{FreshII} where 

\[t\in T_{2} \land t \in R\]

proof is trivial.
\end{comment}

%%%%
To show (\ref{l1wfp2}$\ldotp$ \textsf{IteratorIII}) we can assume for all \[x,tid', o\]
we can assume 
\[ \sigma'.s(x,tid') = o \land \mathsf{iterator} \, tid' \in O'_{2}(o) \land \exists T \ldotp (T,o)  \in F' \]
and we must prove 
\[tid' \in T\]
From assumption ( \ref{l1wfa2}$\ldotp$ \textsf{IteratorIII} ) we know for all \[x,tid',o\] 
\[\sigma.s(x,tid') = o \land \mathsf{iterator} \, tid' \, \in O_{2}(o) \land \exists T \ldotp (T,o) \in F \]
As
\[O_{2} = O'_{2} \land T_{2} = T'_{2}\]
and using 
\[\sigma'.s(x,tid') = o \land \mathsf{iterator} \, tid' \, \in O'_{2}(o) \land \exists T \ldotp (T,o) \in F' \]
and 
all updates in $\sigma$ are done to $tid$ associated variables and \[tid \notin T_{2}\]
implies that 
\[tid' \in T_{2} \land tid' \in T\].

%%%%

To prove (\ref{l1wfp2}$\ldotp$ \textsf{FreshII}) we can assume for some \[ x, tid', o \]
we can assume
\[ \sigma'.s(x,tid') = o \land \mathsf{fresh} \in O_{2}(o) \land tid' \in T_{2} \]
and we must prove 
\[tid' = \sigma'.l\]
From assumption ( \ref{l1wfa2}$\ldotp$ \textsf{FreshII} ) we know 
\begin{gather*}
\forall x,tid',o \ldotp \sigma.s(x,tid') = o \land \mathsf{fresh} \in O_{2}(o) \land tid' \in T_{2} \implies \\
tid' = \sigma.l 
\end{gather*}
As
\[O_{2} = O'_{2} \land T_{2} = T'_{2}\]
and using 
\[ \sigma'.s(x,tid') = o \land \mathsf{fresh} \in O_{2}(o) \land tid' \in T_{2} \]
we conclude that 
\[ \forall x,tid',o \ldotp \sigma.s(x,tid') = o \implies tid' = \sigma.l \]

If we knew $\sigma.s(x,tid') = \sigma'.s(x,tid')$ then we would be done. As we know $\sigma$ and $\sigma'$ only differ on variables associated to $tid$, and moreover $tid \notin T_{2}$, then we know that equality, and thus
\[tid' = \sigma.l\]
as required.

\end{proof}

\begin{Lemma}[Safe MutatorBlock Unlink]
\begin{gather}
\llbracket x.f:=z \rrbracket = \lfloor \llbracket \Gamma,x:\mathsf{rcuNext} \, f \, y, y:\mathsf{rcuNext} \, f \, z \rrbracket_{M,tid} * \{m\}\rfloor  \subseteq \\
\lfloor \llbracket \Gamma,x:\mathsf{rcuNext} \, f \, z, y:\mathsf{unlinked} \rrbracket  * \mathcal{R}(\{m\})\rfloor 
\end{gather}

\end{Lemma}

\begin{proof}

We can  assume 

\begin{gather} \label{l2e2}
(\sigma,O,U,T) \, \in \, \llbracket \Gamma,x:\mathsf{rcuNext} \, f \, y, y:\mathsf{rcuNext} \, f \, z \rrbracket_{M,tid} * \{m\}
\\
\label{l2ewfa}
\textsf{WellFormed}(\sigma,O,U,T)
\end{gather}

And must show that there exists $O',U',T'$ such that

\begin{gather} \label{l2e3}
(  \sigma [ \sigma.s(x,tid),f \mapsto \sigma.s(z,tid) ]  , O', U', T') \in \\
 \llbracket \Gamma,x:\mathsf{rcuNext} \, f \,z, y:\mathsf{unlinked} \rrbracket_{M,tid} * \mathcal{R}(\{m\})
\\
\label{l2ewfa}
\textsf{WellFormed}(  \sigma [ (\sigma.s(x,tid),f \mapsto \sigma.s(z,tid) ] , O', U', T')
\end{gather}

From \ref{l2e2} we can assume 


\begin{gather} \label{l2e4}
(\sigma, O_{1}, U_{1}, T_{1}) \in \llbracket \Gamma, x:\mathsf{rcuNext} \, f \, y, y:\mathsf{rcuNext} \, f  \,z\rrbracket_{M,tid}
\\
 \label{l2e5}
(\sigma, O_{2}, U_{2}, T_{2}) = m 
\\
\label{l2e6}
O_{1} \bullet O_{2} = O
\\
 \label{l2e7}
U_{1} \cup U_{2} = U
\\
 \label{l2e8}
T_{1} \uplus T_{2} = T
\\
\label{l2wfa1}
\textsf{WellFormed}(\sigma,O_{1},U_{1},T_{1})
\\
\label{l2wfa2}
\textsf{WellFormed}(\sigma,O_{2},U_{2},T_{2})
\end{gather}

To prove \ref{l2e3} choose

\[
\begin{array}{cl}
&O' = O[\sigma.s(y,tid) \mapsto O \setminus \mathsf{iterator} \, tid  \cup \mathsf{unlinked}]\\
&U' = U\\
&T' = T
\end{array}
\]

We must prove there exists $O'_{1}, O'_{2}, U'_{1}, U'_{2}, T'_{1}, T'_{2}$ such that


\begin{gather}\label{l2e9}
(\sigma',O'_{1},U'_{1}, T'_{1}) \in \llbracket \Gamma,x:\mathsf{rcuNext}\,f \, z , \, y :\mathsf{unlinked}  \rrbracket_{M,tid}
\\
\label{l2e10}
(\sigma',O'_{2},U'_{2}, T'_{2}) \in \mathcal{R}(\{m\})
\\
 \label{l2e11}
O'_{1} \bullet O'_{2} = O'
\\
 \label{l2e12}
U'_{1} \cup U'_{2} = U' 
\\
 \label{l2e13}
T'_{1} \uplus T'_{2} = T
\\
\label{l2wfp1}
\textsf{WellFormed}(\sigma',O'_{1},U'_{1},T'_{1})
\\
\label{l2wfp2}
\textsf{WellFormed}(\sigma',O'_{2},U'_{2},T'_{2})
\end{gather}

We choose \[O_{2}' = O_{2}, U_{1}' = U_{1}, U_{2}' = U_{2}, T_{1}' = T_{1}, T_{2}' = T_{2}\]  so \ref{l2e12} follows from \ref{l2e7} and \ref{l2e13} follows from \ref{l2e8}

In the rest of the proof, it is useful to know that 

\begin{equation} \label{l2e14}
T_{1} = \{tid\}
\end{equation}
which follows directly from \ref{l2e4}.


To show \ref{l2e11}, pick

\[O_{1}' = O_{1}[\sigma.s(y,tid) \mapsto O_{1} \setminus \textsf{iterator} \, tid \cup \textsf{unlinked}]\]
$O'_{1} \bullet O'_{2} = O'$, follows, as we know $O'_{2}$ does not include $tid$.


From \ref{l2e4} we can assume


\begin{gather} \label{l2e14}
(\sigma,\hat{O}'_{1},U_{1},T_{1})  \in \llbracket \Gamma \rrbracket_{M,tid}
\\
 \label{l2e15}
(\sigma,\hat{O}''_{1},U_{1},T_{1})  \in \llbracket x:\mathsf{rcuNext} \, f \, y \rrbracket_{M,tid}
\\
\label{l2e16}
(\sigma,\hat{O}'''_{1},U_{1},T_{1}) \in \llbracket y:\mathsf{rcuNext} \, f \, z \rrbracket_{M,tid}
\\
 \label{l2e17}
\hat{O}'_{1} \bullet \hat{O}''_{1} \bullet \hat{O}'''_{1} = O_{1}
\end{gather}


To show \ref{l2e9}, we need to show that there exists  $O''_{1}$, $O'''_{1}$, $O''''_{1}$ such that 

\begin{gather} \label{l2e18}
(\sigma', O''_{1},U ,T) \in \llbracket \Gamma \rrbracket_{M,tid}
\\
 \label{l2e19}
(\sigma', O'''_{1},U ,T) \in \llbracket x:\mathsf{rcuNext} \, f \, z \rrbracket_{M,tid}
\\
 \label{l2e20}
(\sigma', O''''_{1},U ,T) \in \llbracket y:\mathsf{unlinked} \rrbracket_{M,tid}
\\ 
\label{l2e21}
O''_{1} \bullet O'''_{1} \bullet O''''_{1} = O'_{1}
\end{gather}

Choosing $\hat{O}'_{1} = O''_{1}$, \ref{l2e18} follows from \ref{l2e14}.

By choosing $O''''_{1}$ to contain the unlinked node and choosing $O'''_{1}$ to contain the iterator from ${O}'_{1}$, we show \ref{l2e21} where \ref{l2e9} follows from.

To show \ref{l2e10} we need to show that 

\[
(\sigma, O_{2}, U_{2}, T_{2}) \mathcal{R} (\sigma', O_{2}, U_{2}, T_{2})
\]
%$\sigma$ agrees with $\sigma'$ on the heap and the stack for threads other than $tid$. By \ref{l1e8} and \ref{l1e14}, $tid \notin T_{2}$ so this is allowed.
which by definition means we must show
\begin{gather}
\label{l2re1}
\sigma.l \in T_{2} \rightarrow \sigma.h = \sigma'.h
\\
\label{l2re2}
\textsf{iterator} \, tid \in O_{2}(o) \rightarrow  o \in dom(\sigma'.h)
\\
\label{l2re3}
O_{2} = O_{2}
\\
\label{l2re4}
U_{2} = U_{2}
\\
\label{l2re5}
T_{2} = T_{2}
\\
\label{l2re6}
t \in T_{2} \rightarrow \forall x \ldotp \sigma.s(x,t) = \sigma'.s(x,t)
\end{gather}


\ref{l2re3}-\ref{l2re5} holds trivially. We know that 
\[T_{2} \subseteq R\]
so \ref{l2re1} holds trivially.

\ref{l2re2} holds as we know from \textsf{\textbf{MustBeAllocated}} that $o \in dom(\sigma.h)$ and thus $o \in dom(\sigma'.h)$. To show \ref{l2re6} we know $T_{1} = \{tid\} \land tid \notin T_{2}$.  We can assume $t \in T_{2} \land t \ne tid$ and must prove that 
\[
\forall x \ldotp \sigma.s(x,t) = \sigma'.s(x,t) 
\]
which holds by the definition of $\sigma'$ as only $tid$s variables could be updated. 


To show \ref{l2wfp1} we assume \ref{l2wfa1} and pick

\[O_{1}' = O_{1}[\sigma.s(y,tid) \mapsto O_{1} \setminus \mathsf{iterator} \, tid \cup \mathsf{unlinked}] \]

In addition, we know  that 

\[T_{1}=\{tid\} \land tid= l\]
and  assume that 
\[o = \sigma'.s(y,tid)\]

(\ref{l2wfp1}.\textsf{IteratorI}) holds as 

\[ l =  tid \land \textsf{unlinked} \in O'_{1}(o)\]

(\ref{l2wfp1}.\textsf{IteratorII}) is trivial.

 To show (\ref{l2wfp1}.\textsf{IteratorIII}), we know that 

\[F \subseteq R\land tid \notin R\] implies that ( \ref{l2wfp1}.\textsf{IteratorIII}) is trivial.

To show (\ref{l2wfp1}.\textsf{UnlinkedI}), we can assume that 
\[ \sigma'.s(y,tid) = o \land \mathsf{unlinked} \in O'_{1}(o) \]
and we show that 
\[\forall o',f' \ldotp  h(o',f') = o \implies \mathsf{unlinked} \in O'_{1}(o') \lor ( \_ ,o') \in F\]
As 
\[\mathsf{unlinked} \in O'_{1}(\sigma'.s(y,tid)) \]
(\ref{l2wfp1}.\textsf{UnlinkedI}) holds.

(\ref{l2wfp1}.\textsf{UnlinkedII-III}) is trivial.

(\ref{l2wfp1}.\textsf{FreshI-II}), (\ref{l2wfp1}.\textsf{WriterNotReader}), (\ref{l2wfp1}.\textsf{MustBeAllocated}) is trivial.

To show (\ref{l2wfp1}.\textsf{FreeingI}) ...To show (\ref{l2wfp1}.\textsf{FreeingII}) ...
%%%%%%%%%%%%%%%%

To show \ref{l2wfp2} we assume \ref{l2wfa2} and we already know that $O_{2}' = O_{2}$,  $U_{2}' = U_{2}$ and $T_{2}' = T_{2}$  .

To prove (\ref{l2wfp2}.\textsf{IteratorI}), for some 
\[ x, o, tid' \]
we assume
\[
\sigma'.s(x,tid') = o \land (x,tid') \notin U'_{2} \land tid' \in T'_{2}
\]
and it is enough to prove 
\[ \mathsf{iterator} \, tid \in O'_{2} \]
as 
\[tid' \in R\]
From assumption (\ref{l2wfp1}.\textsf{IteratorI}) we know
\[ \sigma.s(x,tid') = o \land (x,tid') \notin U_{2} \land tid' \in T_{2} \implies \mathsf{iterator} \, tid \in O_{2}\]
As
\[  O_{2}' = O_{2}  \land U_{2}' = U_{2} \land T_{2}' = T_{2} \]
and using
\[
\sigma'.s(x,tid') = o \land (x,tid') \notin U'_{2} \land tid' \in T'_{2}
\]
and as heap changes, we can conlclude that 
\[\forall o,x \ldotp \sigma'.s(x,tid') = o \land (x,tid') \notin U'_{2} \land tid' \in T'_{2} \implies \mathsf{iterator} tid' \]

Proving (\ref{l2wfp2}.\textsf{IteratorII}) is trivial.


To prove (\ref{l2wfp2}.\textsf{IteratorIII}),  for some
\[o,tid' \] where 
\[tid' \in R\]
we assume 
\[ \mathsf{iterator} \, tid' \in O'_{2} \land \exists T \ldotp (T,o) \in  F \] to prove
\[tid' \in T\]
From assumption  (\ref{l2wfa2}.\textsf{IteratorIII}), we know that 
\[\mathsf{iterator} \, tid' \in O_{2} \land \exists T \ldotp (T,o) \in  F  \implies tid' \in T \]
and using 
\[O_{2} = O'_{2} \land T_{2} = T'_{2}\] with
\[ \mathsf{iterator} \, tid' \in O'_{2} \land \exists T \ldotp (T,o) \in  F \]
enables us to conclude that 
\[tid' \in T\]

To prove (\ref{l2wfp2}.\textsf{Unlinked I}) for some $o$ we can assume
\[\mathsf{unlinked} \in O'_{2}(o) \] and 
\[\forall o',f' \ldotp \sigma'.h(o',f')= o\] to prove
\[ \mathsf{unlinked} \in O'_{2}(o) \lor (\_,o') \in F\]
From assumption (\ref{l2wfa2}.\textsf{Unlinked I})  we know that 
\begin{gather}\mathsf{unlinked} \in O_{2}(o) \text{then} \forall o',f' \ldotp \sigma.h(o',f') = o \implies \\ 
\mathsf{unlinked} \in O_{2}(o') \lor (\_,o') \in F \end{gather}
Using 
\[O_{2} = O'_{2} \land T_{2} = T'_{2}\] and 
\[\mathsf{unlinked} \in O'_{2}(o) \]
\[\forall o',f' \ldotp \sigma'.h(o',f')= o\] 
we can conclude that 
\[ \mathsf{unlinked} \in O'_{2}(o) \lor (\_,o') \in F\]

Proving (\ref{l2wfp2}.\textsf{UnlinkedII-III}) is trivial.

To prove (\ref{l2wfp2}.\textsf{FreshI}) we assume for some node $o$
\[\mathsf{fresh} \in O'_{2}(o) \]
\[\forall o',f'  \ldotp \sigma'.h(o',f') = o\]
to prove
\[ \mathsf{fresh} \in O'_{2}(o')\]
From assumption (\ref{l2wfa2}.\textsf{FreshI}), we know that 
\[ \forall o \ldotp \mathsf{fresh} \in O_{2}(o) \text{then} \forall o',f' \ldotp \sigma.h(o',f') = o \implies \mathsf{fresh} \in O_{2}(o') \]
and using 
\[ O_{2} = O'_{2} \land T_{2} = T'_{2} \] with 
\[ \mathsf{fresh} \in O'_{2}(o) \]
\[ \forall o',f'  \ldotp \sigma'.h(o',f') = o\]
we conclude that 
\[fresh \in O'_{2}(o')\]

Proving (\ref{l2wfp2}.\textsf{FreshII}), (\ref{l2wfp2}.\textsf{FreeingI-II}) is trivial.

\end{proof}
\newcommand{\newln}{\\&\quad\quad{}} 


\begin{Lemma}[Safe MutatorBlock Link]
\begin{gather}
\llbracket x.f:=z \rrbracket \lfloor \llbracket \Gamma,x:\mathsf{rcuNext} \, f \, y \, , z:\mathsf{fresh} \, f \, y \rrbracket_{\textsf{M},tid} * \{m\}\rfloor  \subseteq \\ 
\lfloor \llbracket \Gamma,x:\mathsf{rcuNext} \, f \, z \, , z:\mathsf{rcuNext} \, f \, y  \rrbracket_{\textsf{M},tid}  * \mathcal{R}(\{m\})\rfloor 
\end{gather}
\end{Lemma}
\begin{proof}
We assume 
\begin{gather} \label{l3e1}
(\sigma,O,U,T) \, \in \, \llbracket \Gamma,x:\mathsf{rcuNext} \, f \, y \, , z:\mathsf{fresh} \, f \, y \rrbracket_{\textsf{M},tid} * \{m\}
\\
\label{l3ewfa}
\textsf{WellFormed}(\sigma,O,U,T)\\
\sigma.h(\sigma.s(x,tid),f) = \sigma.s(y,tid) 
\end{gather}
And must show that there exists $O',U',T'$ such that
\begin{gather} 
\begin{split}
\label{l3e3} 
(\sigma[ \sigma.s(x,tid),f \mapsto\sigma.s(z,tid) ], O',U',T')  \\
\in \llbracket \Gamma,x:\mathsf{rcuNext} \, f \, z \, , z:\mathsf{rcuNext} \, f \, y \rrbracket_{\textsf{M},tid} * \mathcal{R}(\{m\})
\end{split}
\\
\label{l3ewfa}
\textsf{WellFormed}(\sigma[ \sigma.s(x,tid),f \mapsto\sigma.s(z,tid)],O',U',T')
\end{gather}

From \ref{l3e1} we can assume 

\begin{gather} \label{l3e4}
(\sigma, O_{1}, U_{1}, T_{1}) \in \llbracket \Gamma, x:\mathsf{rcuNext} \, f \, y, z:\mathsf{fresh} \, f \, y\rrbracket_{M,tid}
\\
 \label{l3e5}
(\sigma, O_{2}, U_{2}, T_{2}) = m 
\\
\label{l3e6}
O_{1} \bullet O_{2} = O
\\
 \label{l3e7}
U_{1} \cup U_{2} = U
\\
 \label{l3e8}
T_{1} \uplus T_{2} = T
\\
\label{l3wfa1}
\textsf{WellFormed}(\sigma,O_{1},U_{1},T_{1})
\\
\label{l3wfa2}
\textsf{WellFormed}(\sigma,O_{2},U_{2},T_{2})
\end{gather}

To prove \ref{l3e3} choose

\[
\begin{array}{cl}
&O' = O[\sigma.s(z,tid) \mapsto O  \setminus \mathsf{fresh}  \cup \mathsf{iterator} \, tid]\\
&U' = U\\
&T' = T
\end{array}
\]

We must prove there exists $O'_{1}, O'_{2}, U'_{1}, U'_{2}, T'_{1}, T'_{2}$ such that


\begin{gather}\label{l3e9}
(\sigma',O'_{1},U'_{1}, T'_{1}) \in \llbracket \Gamma,x:\mathsf{rcuNext}\,f \, z , \, z:\mathsf{rcuNext}\,f \, y  \rrbracket_{M,tid}
\\
\label{l3e10}
(\sigma',O'_{2},U'_{2}, T'_{2}) \in \mathcal{R}(\{m\})
\\
 \label{l3e11}
O'_{1} \bullet O'_{2} = O'
\\
 \label{l3e12}
U'_{1} \cup U'_{2} = U' 
\\
 \label{l3e13}
T'_{1} \uplus T'_{2} = T
\\
\label{l3wfp1}
\textsf{WellFormed}(\sigma',O'_{1},U'_{1},T'_{1})
\\
\label{l3wfp2}
\textsf{WellFormed}(\sigma',O'_{2},U'_{2},T'_{2})
\end{gather}

We choose 

\[O'_{2} = O_{2}, U'_{1} = U_{1}, U'_{2} = U_{2}, T'_{1} = T_{1}, T'_{2} = T_{2}  \] so \ref{l3e12} follows from \ref{l3e7} and \ref{l3e13} follows from \ref{l3e8}.

In the rest of the proof , it is useful to knowthat
\begin{equation} \label{l3e14}
T_{1} = \{tid\}
\end{equation}
which follows directly from \ref{l3e4}.

To show \ref{l3e11}, pick

\[O_{1}' = O_{1}[\sigma.s(z,tid) \mapsto O_{1} \setminus \textsf{fresh}  \cup \textsf{iterator} \, tid]\]
$O'_{1} \bullet O'_{2} = O'$, follows, as we know $O'_{2}$ does not include $tid$.


From \ref{l3e4} we can assume

\begin{gather} \label{l3e14}
(\sigma,\hat{O}'_{1},U_{1},T_{1})  \in \llbracket \Gamma \rrbracket_{M,tid}
\\
 \label{l3e15}
(\sigma,\hat{O}''_{1},U_{1},T_{1})  \in \llbracket x:\mathsf{rcuNext} \, f \, y \rrbracket_{M,tid}
\\
\label{l3e16}
(\sigma,\hat{O}'''_{1},U_{1},T_{1}) \in \llbracket z:\mathsf{fresh} \, f \, y \rrbracket_{M,tid}
\\
 \label{l3e17}
\hat{O}'_{1} \bullet \hat{O}''_{1} \bullet \hat{O}'''_{1} = O_{1}
\end{gather}


To show \ref{l3e9}, we need to show that there exists  $O''_{1}$, $O'''_{1}$, $O''''_{1}$ such that 

\begin{gather} \label{l3e18}
(\sigma', O''_{1},U ,T) \in \llbracket \Gamma \rrbracket_{M,tid}
\\
 \label{l3e19}
(\sigma', O'''_{1},U ,T) \in \llbracket x:\mathsf{rcuNext} \, f \, z \rrbracket_{M,tid}
\\
 \label{l3e20}
(\sigma', O''''_{1},U ,T) \in \llbracket z:\mathsf{rcuNext} \,f \, y \rrbracket_{M,tid}
\\ 
\label{l3e21}
O''_{1} \bullet O'''_{1} \bullet O''''_{1} = O'_{1}
\end{gather}

Choosing $\hat{O}'_{1} = O''_{1}$, \ref{l3e18} follows from \ref{l3e14}.

By choosing $O''''_{1}$ and $O'''_{1}$ to contain the iterators from ${O}'_{1}$, we show \ref{l3e21} where \ref{l3e9} follows from.

To show \ref{l3e10} we need to show that 

\[
(\sigma, O_{2}, U_{2}, T_{2}) \mathcal{R} (\sigma', O_{2}, U_{2}, T_{2})
\]
%$\sigma$ agrees with $\sigma'$ on the heap and the stack for threads other than $tid$. By \ref{l1e8} and \ref{l1e14}, $tid \notin T_{2}$ so this is allowed.
which by definition means we must show
\begin{gather}
\label{l3re1}
\sigma.l \in T_{2} \rightarrow \sigma.h = \sigma'.h
\\
\label{l3re2}
\textsf{iterator} \, tid \in O_{2}(o) \rightarrow  o \in dom(\sigma'.h)
\\
\label{l3re3}
O_{2} = O_{2}
\\
\label{l3re4}
U_{2} = U_{2}
\\
\label{l3re5}
T_{2} = T_{2}
\\
\label{l3re6}
t \in T_{2} \rightarrow \forall x \ldotp \sigma.s(x,t) = \sigma'.s(x,t)
\end{gather}

\ref{l3re3}-\ref{l3re5} holds trivially. We know that 
\[T_{2} \subseteq R\]
so \ref{l3re1} holds trivially.

\ref{l3re2} holds as we know from \textsf{\textbf{MustBeAllocated}} that $o \in dom(\sigma.h)$ and thus $o \in dom(\sigma'.h)$. To show \ref{l3re6} we know $T_{1} = \{tid\} \land tid \notin T_{2}$.  We can assume $t \in T_{2} \land t \ne tid$ and must prove that 
\[
\forall x \ldotp \sigma.s(x,t) = \sigma'.s(x,t) 
\]
which holds by the definition of $\sigma'$ as only $tid$s variables could be updated. 

To prove \ref{l3wfp1} we know 
\[O_{1}' = O_{1}[\sigma.s(z,tid) \mapsto O_{1} \setminus \mathsf{fresh}  \cup \mathsf{iterator} \, tid ] \]
in addition to
\[T_{1}=\{tid\} \land tid= l\]

(\ref{l3wfp1}.\textsf{IteratorI}) holds as 

\[ l =  tid \] for
\[ o' = \sigma'.s(x,tid) \land o'' = \sigma'.s(z,tid) \] with knowing that
\[O'_{1} = O_{1}[\sigma.s(z,tid) \mapsto O_{1}  \setminus \mathsf{fresh}  \cup \mathsf{iterator} \, tid]\]
we can conclude that (\ref{l3wfp1}.\textsf{IteratorI}) holds as  
\[ \textsf{iterator} \, tid \in O'_{1}(o') \land \textsf{iterator} \, tid \in O'_{1}(o'') \]

Proving (\ref{l3wfp1}.\textsf{IteratorII})  is trivial.
 To show (\ref{l3wfp1}.\textsf{IteratorIII}), we know that 

Knowing that \[F \subseteq R\land tid \notin R\] makes ( \ref{l3wfp1}.\textsf{IteratorIII}) is trivial.

Proving (\ref{l3wfp1}.\textsf{Unlinked I}) is trivial.

To prove (\ref{l3wfp1}.\textsf{Unlinked II}) we can assume 
\begin{gather} \forall o \ldotp (T,o) \in F' \implies (\forall o',f' \ldotp h(o',f') = o \implies (T',o') \in F \land T' \subseteq T) \end{gather}

From (\ref{l3wfa1}.\textsf{Unlinked II}), we know that 
\begin{gather} \forall o \ldotp (T,o) \in F \implies (\forall o',f' \ldotp h(o',f') = o \implies (T',o') \in F and T' \subseteq T) \end{gather}
As we know that  $O_{1}$  is updated with a node $o'$ such that 
 \[o=\sigma'.s(y,tid) \land  o' = \sigma'.s(z,tid)   \]
\[z:\mathsf{rcuNext}\,f \, y \land \mathsf{iterator} \, tid \in O'_{1}(o') \land o' \in dom(\sigma'.h) \]
and 
\[tid \notin R \]
we can conclude that 
\[ (\forall o',f' \ldotp \sigma'.h(o',f') = o \implies (T',o') \in F and T' \subseteq T) \]

From assumption of (\ref{l3wfa1}.\textsf{UnlinkedIII}), we know that for some $o$ 
\[ \mathsf{iterator} \; l \in O_{1}(o) \implies \mathsf{unlinked} \notin O_{1}(o)\]
As 
\[ T_{1} = T'_{1} \land tid = l \] and from our assumption from (\ref{l3wfp1}.\textsf{UnlinkedIII})
\[\mathsf{itearator} \; l \in O'_{1}(o) \] and
\[ O_{1}' = O_{1}[\sigma.s(z,tid) \mapsto O_{1} \setminus \mathsf{fresh}  \cup \mathsf{iterator} \, tid ] \] 
we can conclude that 
\[ \mathsf{unlinked} \notin O'_{1} \]  

(\ref{l3wfp1}.\textsf{Fresh I-II}) is trivial.

(\ref{l3wfp1}.\textsf{FreeingI-II, MustbeAllocated, WriterNotReader}) is trivial.
For proof of \ref{l3wfp2}, we know 
\[ tid' \in T_{2} \land T_{2} = T'_{2} \land tid' \subseteq R \land  O_{2} = O'_{2} \land U_{2}  = U'_{2} \]
(\ref{l3wfp2}.\textsf{Ownership}) holds as we know that heap owns all locations that it owns before. 

(\ref{l3wfp2}.\textsf{IteratorI}) holds as from (\ref{l3wfa2}.\textsf{IteratorI}), we assume
\begin{gather*} \forall o \ldotp o \in \mathsf{RCU} \land \sigma.s(x,tid') = o \land (x,tid') \notin U \implies \\
\mathsf{iterator} \, tid' \in O_{2}(o)\end{gather*}From (\ref{l3wfp2}.\textsf{IteratorI}), we can assume that 
\[\forall o \ldotp o \in \mathsf{RCU} \land \sigma'.s(x,tid') = o \land (x,tid') \notin U\] and conclude that 
\[\mathsf{iterator} \, tid' \in O'_{2}(o)\]

(\ref{l3wfp2}.\textsf{IteratorII}) holds as from (\ref{l3wfa2}.\textsf{IteratorII}), we assume
\begin{gather} \forall tid' \in R , o,o' \ldotp \mathsf{iterator} \, tid' \in O_{2}(o) \land \mathsf{iterator} \, tid' \in O_{2}(o') \implies \\
 o=o' \end{gather}
As we know that 
\[O_{2} = O'_{2}\] which makes showing
(\ref{l3wfp2}.\textsf{IteratorII}) trivial.

(\ref{l3wfp2}.\textsf{IteratorIII}) holds as from (\ref{l3wfa2}.\textsf{IteratorIII}), we assume for some $o,tid'$
 \[ \mathsf{iterator} \, tid' \in O_{2}(o) \land \exists T \ldotp (T,o) \in F \implies tid' \in T_{2}\] and we can assume
\[\mathsf{iterator} \, tid' \in O'_{2} \land \exists T \ldotp (T,o) \in F'\] to prove
\[tid' \in T'_{2}\]

As \[ O_{2} = O'_{2} \land T_{2} = T'_{2} \] and using 
\[ \mathsf{iterator} \, tid'  \in O'_{2} \land \exists T \ldotp (T,o) \in F' \] we can conclude that $(T,o)$ stays same because 
\[T \subseteq R\land T \subseteq T'_{2}\] and 
(\ref{l3wfp2}.\textsf{IteratorIII}) holds.

(\ref{l3wfp2}.\textsf{Unlinked I-III-IV}) is trivial.

Proving (\ref{l3wfp2}.\textsf{Unlinked II}) 

Proving (\ref{l3wfp2}.\textsf{Fresh I}) is trivial.

To prove (\ref{l3wfp2}.\textsf{Fresh II})  we can  assume for some $x,tid',o$
\[\sigma'.s(x,tid') = o \land \mathsf{fresh} \in O'_{2}(o) \land tid' \in T'_{2}\]
and must prove 
\[tid' = \sigma'.l\]
From assumption (\ref{l3wfa2}.\textsf{Fresh II}) we know 
\[\forall x,tid',o \ldotp \sigma.s(x,tid') = o \land \mathsf{fresh} \in O_{2}(o) \land tid' \in T_{2} \implies tid' = \sigma.l\]
As we know \[O_{2} = O'_{2} \land T_{2} = T'_{2}\] and \[\sigma'.s(x,tid') = o \land \mathsf{fresh} \in O'_{2}(o) \land tid' \in T'_{2}\] and because $\sigma$ and $\sigma'$ only differ on  variables related to $tid$ and 
\[tid \notin T'_{2}\] (\ref{l3wfa2}.\textsf{Fresh II}) holds.

Proving (\ref{l3wfp2}.\textsf{MustBeAllocated-FreeingI-II}) is  trivial.

\end{proof}

\begin{Lemma}[Safe MutatorBlock Allocate a Node]
\[
\llbracket y := new \rrbracket \lfloor \llbracket \Gamma,y:\_ \rrbracket_{\textsf{M},tid} * \{m\}\rfloor  \subseteq\lfloor \llbracket \Gamma,y:\mathsf{fresh} \rrbracket_{\textsf{M},tid}  * \mathcal{R}(\{m\})\rfloor 
\]
\end{Lemma}
\begin{proof}
We assume 
\begin{gather} \label{l4e1}
(\sigma,O,U,T) \, \in \, \llbracket \Gamma,y:\_ \rrbracket_{\textsf{M},tid} * \{m\}
\\
\label{l4ewfa}
\textsf{WellFormed}(\sigma,O,U,T) 
\end{gather}
And must show that there exists $O',U',T'$ such that
\begin{gather} 
\begin{split}
\label{l4e3} 
(\sigma[ \sigma.s(y,tid) \mapsto \sigma.h(\sigma.s(y,tid),f) ], O',U',T')  \\
\in \llbracket \Gamma, y:\_ \rrbracket_{\textsf{M},tid} * \mathcal{R}(\{m\})
\end{split}
\\
\label{l4ewfa}
\textsf{WellFormed}(\sigma[ \sigma.s(y,tid) \mapsto \sigma.h(\sigma.s(y,tid),f) ],O',U',T')
\end{gather}

From \ref{l4e1} we can assume 

\begin{gather} \label{l4e4}
(\sigma, O_{1}, U_{1}, T_{1}) \in \llbracket \Gamma, y:\_ \rrbracket_{M,tid}
\\
 \label{l4e5}
(\sigma, O_{2}, U_{2}, T_{2}) = m 
\\
\label{l4e6}
O_{1} \bullet O_{2} = O
\\
 \label{l4e7}
U_{1} \cup U_{2} = U
\\
 \label{l4e8}
T_{1} \uplus T_{2} = T
\\
\label{l4wfa1}
\textsf{WellFormed}(\sigma,O_{1},U_{1},T_{1})
\\
\label{l4wfa2}
\textsf{WellFormed}(\sigma,O_{2},U_{2},T_{2})
\end{gather}

To prove \ref{l4e3} choose

\[
\begin{array}{cl}
&O' = O[\sigma.s(y,tid) \mapsto O  \cup \mathsf{fresh}]\\
&U' = U\\
&T' = T
\end{array}
\]

We must prove there exists $O'_{1}, O'_{2}, U'_{1}, U'_{2}, T'_{1}, T'_{2}$ such that


\begin{gather}\label{l4e9}
(\sigma',O'_{1},U'_{1}, T'_{1}) \in \llbracket \Gamma,x:\mathsf{fresh}  \rrbracket_{M,tid}
\\
\label{l4e10}
(\sigma',O'_{2},U'_{2}, T'_{2}) \in \mathcal{R}(\{m\})
\\
 \label{l4e11}
O'_{1} \bullet O'_{2} = O'
\\
 \label{l4e12}
U'_{1} \cup U'_{2} = U' 
\\
 \label{l4e13}
T'_{1} \uplus T'_{2} = T
\\
\label{l4wfp1}
\textsf{WellFormed}(\sigma',O'_{1},U'_{1},T'_{1})
\\
\label{l4wfp2}
\textsf{WellFormed}(\sigma',O'_{2},U'_{2},T'_{2})
\end{gather}

We choose 

\[O'_{2} = O_{2}, U'_{1} = U_{1}, U'_{2} = U_{2}, T'_{1} = T_{1}, T'_{2} = T_{2}  \] so \ref{l4e12} follows from \ref{l4e7} and \ref{l4e13} follows from \ref{l4e8}.

In the rest of the proof , it is useful to knowthat
\begin{equation} \label{l4e14}
T_{1} = \{tid\}
\end{equation}
which follows directly from \ref{l4e4}.

To show \ref{l4e11}, pick

\[O_{1}' = O_{1}[\sigma.s(y,tid) \mapsto O_{1} \cup \mathsf{fresh} ]\]
$O'_{1} \bullet O'_{2} = O'$, follows, as we know $O'_{2}$ does not include $tid$.


From \ref{l4e4} we can assume

\begin{gather} \label{l4e14}
(\sigma,\hat{O}'_{1},U_{1},T_{1})  \in \llbracket \Gamma \rrbracket_{M,tid}
\\
 \label{l4e15}
(\sigma,\hat{O}''_{1},U_{1},T_{1})  \in \llbracket y:\_ \rrbracket_{M,tid}
\\
 \label{l4e16}
\hat{O}'_{1} \bullet \hat{O}''_{1} = O_{1}
\end{gather}


To show \ref{l4e9}, we need to show that there exists  $O''_{1}$, $O'''_{1}$ such that 

\begin{gather} 	\label{l4e17}
(\sigma', O''_{1},U ,T) \in \llbracket \Gamma \rrbracket_{M,tid}
\\
 \label{l4e18}
(\sigma', O'''_{1},U ,T) \in \llbracket y:\mathsf{fresh} \rrbracket_{M,tid}
\\
\label{l4e19}
O''_{1} \bullet O'''_{1} = O'_{1}
\end{gather}

Choosing $\hat{O}'_{1} = O''_{1}$, \ref{l4e17} follows from \ref{l4e14}.

By choosing $O'''_{1}$ to contain the fresh node from ${O}'_{1}$, we show \ref{l4e19} where \ref{l4e9} follows from.

To show \ref{l4e10} we need to show that 

\[
(\sigma, O_{2}, U_{2}, T_{2}) \mathcal{R} (\sigma', O_{2}, U_{2}, T_{2})
\]
%$\sigma$ agrees with $\sigma'$ on the heap and the stack for threads other than $tid$. By \ref{l1e8} and \ref{l1e14}, $tid \notin T_{2}$ so this is allowed.
which by definition means we must show
\begin{gather}
\label{l4re1}
\sigma.l \in T_{2} \rightarrow \sigma.h = \sigma'.h
\\
\label{l4re2}
\textsf{iterator} \, tid \in O_{2}(o) \rightarrow  o \in dom(\sigma'.h)
\\
\label{l4re3}
O_{2} = O_{2}
\\
\label{l4re4}
U_{2} = U_{2}
\\
\label{l4re5}
T_{2} = T_{2}
\\
\label{l4re6}
t \in T_{2} \rightarrow \forall x \ldotp \sigma.s(x,t) = \sigma'.s(x,t)
\end{gather}

\ref{l4re3}-\ref{l4re5} holds trivially. We know that 
\[T_{2} \subseteq R\]
so \ref{l4re1} holds trivially. \ref{l4re2} holds trivially.

To show \ref{l4re6} we know $T_{1} = \{tid\} \land tid \notin T_{2}$.  We can assume $t \in T_{2} \land t \ne tid$ and must prove that 
\[
\forall x \ldotp \sigma.s(x,t) = \sigma'.s(x,t) 
\]
which holds by the definition of $\sigma'$ as only $tid$s variables could be updated. 

To prove \ref{l4wfp1} we know 
\[O_{1}' = O_{1}[\sigma.s(y,tid) \mapsto O_{1} \cup \mathsf{fresh} ] \]
in addition to
\[T_{1}=\{tid\} \land tid= l\]

To prove (\ref{l4wfp1}.\textsf{Fresh I}) , we know from \textsf{InDomain} that
 \[\forall o,o',f' \ldotp \sigma.h(o',f') = o \implies o' \in dom(h) \] and
as we know that allocating a new is done from free list or some new heap location which is not in domain of rcu data structure before.  This makes (\ref{l4wfp1}.\textsf{Fresh I}) holds.

To prove (\ref{l4wfp1}.\textsf{Fresh II}) we know from (\ref{l4wfa1}.\textsf{Fresh II}) for some $x,o$ 
\[ \sigma.s(x,tid) = o \land \mathsf{fresh} \in O_{1}(o) \implies tid = l \]
and as we know that 
\[ T_{1} = T'_{1} \land tid = l \]
(\ref{l4wfp1}.\textsf{Fresh II} becomes trivial.

Proving \ref{l4wfp2} is trivial
\end{proof}

\begin{Lemma}[Safe MutatorBlock Free an Unlinked Node]
\begin{gather}
\llbracket  \mathsf{delayedfree}(x) \rrbracket \lfloor \llbracket \Gamma,x:\mathsf{unlinked} \rrbracket_{\textsf{M},tid} * \{m\}\rfloor  \subseteq \\ 
\lfloor \llbracket \Gamma,x:\mathsf{undef} \rrbracket_{\textsf{M},tid}  * \mathcal{R}(\{m\})\rfloor 
\end{gather}
\end{Lemma}
\begin{proof}

We assume 
\begin{gather} \label{l5e1}
(\sigma,O,U,T) \, \in \, \llbracket \Gamma,x:\mathsf{unlinked} \rrbracket_{\textsf{M},tid} * \{m\}
\\
\label{l5ewfa}
\textsf{WellFormed}(\sigma,O,U,T) 
\end{gather}
And must show that there exists $O',U',T'$ such that
\begin{gather} 
\begin{split}
\label{l5e3} 
(\sigma[ \sigma.s(x,tid) \mapsto \_ ], O',U',T')  \\
\in \llbracket \Gamma, x:\mathsf{unlinked} \rrbracket_{\textsf{M},tid} * \mathcal{R}(\{m\})
\end{split}
\\
\label{l5ewfa}
\textsf{WellFormed}(\sigma[ \sigma.s(x,tid) \mapsto \_) ],O',U',T')
\end{gather}

From \ref{l5e1} we can assume 

\begin{gather} \label{l5e4}
(\sigma, O_{1}, U_{1}, T_{1}) \in \llbracket \Gamma, y:\mathsf{unlinked} \rrbracket_{M,tid}
\\
 \label{l5e5}
(\sigma, O_{2}, U_{2}, T_{2}) = m 
\\
\label{l5e6}
O_{1} \bullet O_{2} = O
\\
 \label{l5e7}
U_{1} \cup U_{2} = U
\\
 \label{l5e8}
T_{1} \uplus T_{2} = T
\\
\label{l5wfa1}
\textsf{WellFormed}(\sigma,O_{1},U_{1},T_{1})
\\
\label{l5wfa2}
\textsf{WellFormed}(\sigma,O_{2},U_{2},T_{2})
\end{gather}

To prove \ref{l5e3} choose

\[
\begin{array}{cl}
&O' = O[\sigma.s(x,tid) \mapsto O  \setminus \mathsf{unlinked} \cup \mathsf{undef}]\\
&U' = U[x \mapsto U \cup \mathsf{undef}]\\
&T' = T
\end{array}
\]

We must prove there exists $O'_{1}, O'_{2}, U'_{1}, U'_{2}, T'_{1}, T'_{2}$ such that


\begin{gather}\label{l5e9}
(\sigma',O'_{1},U'_{1}, T'_{1}) \in \llbracket \Gamma,x:\mathsf{undef}  \rrbracket_{M,tid}
\\
\label{l5e10}
(\sigma',O'_{2},U'_{2}, T'_{2}) \in \mathcal{R}(\{m\})
\\
 \label{l5e11}
O'_{1} \bullet O'_{2} = O'
\\
 \label{l5e12}
U'_{1} \cup U'_{2} = U' 
\\
 \label{l5e13}
T'_{1} \uplus T'_{2} = T
\\
\label{l5wfp1}
\textsf{WellFormed}(\sigma',O'_{1},U'_{1},T'_{1})
\\
\label{l5wfp2}
\textsf{WellFormed}(\sigma',O'_{2},U'_{2},T'_{2})
\end{gather}

We choose 

\[O'_{2} = O_{2}, U'_{2} = U_{2}, T'_{1} = T_{1}, T'_{2} = T_{2}  \] so \ref{l5e12} follows from \ref{l5e7} and \ref{l5e13} follows from \ref{l5e8}.

In the rest of the proof , it is useful to knowthat
\begin{equation} \label{l5e14}
T_{1} = \{tid\}
\end{equation}
which follows directly from \ref{l5e4}.

To show \ref{l5e11}, pick

\[O_{1}' = O_{1}[\sigma.s(x,tid) \mapsto O_{1} \setminus \mathsf{unlinked} \cup \mathsf{undef} ]\]
$O'_{1} \bullet O'_{2} = O'$, follows, as we know $O'_{2}$ does not include $tid$.

And pick

\[U_{1}' = U_{1}[x \mapsto U \cup \mathsf{undef}]\]
$U'_{1} \cup U'_{2} = U'$, follows, as we know $U'_{2}$ does not include $tid$.

From \ref{l5e4} we can assume

\begin{gather} \label{l5e14}
(\sigma,\hat{O}'_{1},U_{1},T_{1})  \in \llbracket \Gamma \rrbracket_{M,tid}
\\
 \label{l5e15}
(\sigma,\hat{O}''_{1},U_{1},T_{1})  \in \llbracket x:\mathsf{unlinked} \rrbracket_{M,tid}
\\
 \label{l5e16}
\hat{O}'_{1} \bullet \hat{O}''_{1} = O_{1}
\end{gather}


To show \ref{l5e9}, we need to show that there exists  $O''_{1}$, $O'''_{1}$ such that 

\begin{gather} 	\label{l5e17}
(\sigma', O''_{1},U ,T) \in \llbracket \Gamma \rrbracket_{M,tid}
\\
 \label{l5e18}
(\sigma', O'''_{1},U ,T) \in \llbracket y:\mathsf{undef} \rrbracket_{M,tid}
\\
\label{l5e19}
O''_{1} \bullet O'''_{1} = O'_{1}
\end{gather}

Choosing $\hat{O}'_{1} = O''_{1}$, \ref{l5e17} follows from \ref{l5e14}.

By choosing $O'''_{1}$ to contain the undefined node from ${O}'_{1}$, we show \ref{l5e19} where \ref{l5e9} follows from.

To show \ref{l5e10} we need to show that 

\[
(\sigma, O_{2}, U_{2}, T_{2}) \mathcal{R} (\sigma', O_{2}, U_{2}, T_{2})
\]
%$\sigma$ agrees with $\sigma'$ on the heap and the stack for threads other than $tid$. By \ref{l1e8} and \ref{l1e14}, $tid \notin T_{2}$ so this is allowed.
which by definition means that we must show
\begin{gather}
\label{l5re1}
\sigma.l \in T_{2} \rightarrow \sigma.h = \sigma'.h
\\
\label{l5re2}
\textsf{iterator} \, tid \in O_{2}(o) \rightarrow  o \in dom(\sigma'.h)
\\
\label{l5re3}
O_{2} = O_{2}
\\
\label{l5re4}
U_{2} = U_{2}
\\
\label{l5re5}
T_{2} = T_{2}
\\
\label{l5re6}
t \in T_{2} \rightarrow \forall x \ldotp \sigma.s(x,t) = \sigma'.s(x,t)
\end{gather}

\ref{l5re3}-\ref{l5re5} holds trivially. We know that 
\[T_{2} \subseteq R\]
so \ref{l5re1} holds trivially. \ref{l5re2} holds trivially.

To show \ref{l5re6} we know $T_{1} = \{tid\} \land tid \notin T_{2}$.  We can assume $t \in T_{2} \land t \ne tid$ and must prove that 
\[
\forall x \ldotp \sigma.s(x,t) = \sigma'.s(x,t) 
\]
which holds by the definition of $\sigma'$ as only $tid$s variables could be updated. 

To prove \ref{l5wfp1} and we know \ref{l5wfp2}
\[O_{1}' = O_{1}[\sigma.s(x,tid) \mapsto O_{1} \setminus \mathsf{unlinked} \cup \mathsf{undef} ] \]
\[T_{1}=\{tid\} \land tid= l\]
\[O_{2} = O'_{2}\]
Proving \ref{l5wfp1} and \ref{l5wfp2} is trivial.
\end{proof}

