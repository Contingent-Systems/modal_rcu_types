%!TEX root = ./paper.tex
\section{Soundness}
\label{sec:meminv}
To prove soundness, we define dynamic language semantics and relate with typing rules. We relate the type rules to the semantics by defining a denotation of type environments in terms of an extended machine state.

We define logic states, $\textsf{LState}$, ranged over by $\Sigma$ to be 
\begin{itemize}
\item A machine state, $\sigma$; 
\item An observation map, O, of type $ \textsf{Loc} \to \mathcal{P}(\textsf{obs})$
\item Undefined variable map, $U$, of type $\mathcal{P}(\textsf{Var})$
\item Set of threads, $T$, of type $\mathcal{P}(\textsf{TIDS})$
\end{itemize}
Reification is then trivially given by the first component.

Each memory region can be observed in the following types in a snapshot taken at any time.
\[
\textsf{obs} := \texttt{iterator} \; \mathrm{tid} \mid
			    \texttt{unlinked} \mid
			    \texttt{fresh}
\]

[Perhaps add a toBeFreed observation, that accounts for the time after the deferred free call, before it is actually freed.]

\[
\begin{array}{cl}
\mathcal{M}_{RCU} \stackrel{def}{=} &\{ m \in \textsf{MState} \times O \times U \times T \mid  \textsf{WellFormed}( m) \}
\end{array}
\]
and we define the set of objects that are involved in an RCU datastructure as:
\[
\textsf{RCU} =  
\{ o \mid \textsf{ftype}(f) = \textsf{rcu} \land \exists o' \ldotp h(o',f) = o \}
\]
where
\[
\begin{array}{cl}
\textsf{WellFormed}(\sigma,O,U,T) \stackrel{def}{=} 		&\textbf{\textsf{Ownership}}(\sigma,O,U,T)\\
								&\land$ $\textbf{\textsf{Iterator I}}(\sigma,O,U,T)\\
								&\land$ $\textbf{\textsf{Iterator II}}(\sigma,O,U,T)\\
								&\land$ $\textbf{\textsf{Iterator III}}(\sigma,O,U,T)\\
								&\land$ $\textbf{\textsf{Unlinked I}}(\sigma,O,U,T)\\
								&\land$ $\textbf{\textsf{Unlinked II}}(\sigma,O,U,T)\\	
								&\land$ $\textbf{\textsf{Unlinked III}}(\sigma,O,U,T)\\					
								&\land$ $\textbf{\textsf{Fresh I}}(\sigma,O,U,T)\\
								&\land$ $\textbf{\textsf{Fresh II}}(\sigma,O,U,T)\\
								&\land$ $\textbf{\textsf{Writer Not Reader}}(\sigma,O,U,T)\\
								&\land$ $\textbf{\textsf{Freeing I}}(\sigma,O,U,T)\\
								&\land$ $\textbf{\textsf{Freeing II}}(\sigma,O,U,T)\\
								&\land$ $\textbf{\textsf{MustBeAllocated}}(\sigma,O,U,T)\\
								&\land$ $\textbf{\textsf{InDomain}}(\sigma,O,U,T)\\
\end{array}
\]

\begin{figure*}[!htbp]
\footnotesize

\[
\textbf{IdenticalSequences}((f_1 \ldots f_n), (f'_1 \ldots f'_m) ) =
\left\{
\begin{array}{cl}
	& n = m \\
	& \land \forall_{1 \leq i \leq n} \ldotp \textsf{ftype}(f_i) = \textsf{ftype}(f'_i) = \textsf{rcu} \\
	& \land  f_i = f'_i 
\end{array}
\right\}
\]

\[
\textbf{h$^{*}$}(o = \textsf{UniqueRoot}, (f_1 \ldots f_n)) = 
\left\{
\begin{array}{cl}
	&\textbf{h$^{*}$}_{0}(o,(f_1 \ldots f_n)) \\
	&\textbf{h$^{*}$}_{1}( h(o,f_1), (f_2 \ldots f_n))   \\ 
	&\textbf{h$^{*}$}_{2}(h( h(o,f_1), f_2), (f_3 \ldots f_n))  \\
	&\ldotp \\ 
	&\ldotp \\
	&\textbf{h$^{*}$}_{n-1}(h(\ldots(h(o,f_1),\ldots, f_{n-1}),f_n)\\
	&\textbf{h$^{*}$}_{n}(h(\ldots(h(o,f_1),\ldots, f_{n}), \_)
\end{array}
\right\}
\]
\[
\textbf{PathsToNode}(o = \textsf{UniqueRoot} , o') = 
\left\{ (f_1 \ldots f_n) | \textbf{h$^{*}$}(o,f_1 \ldots f_n ) = o' \right\}
\]
\[
\textbf{IsSingleton}(S = \{\ldots\} ) = \mid S \mid = 1
\]

\caption{Utility Definitions for Memory Invariants}
\label{fig:definitionWellFormed-Utility}
\end{figure*}

\begin{figure*}[!htbp]
\footnotesize
\[
\textbf{Ownership}(\sigma,O,U,T) =
\left\{
\begin{array}{cl}
 	 & \forall f,o,o',f' \ldotp \\
	& \sigma.h(o,f) = v \land \sigma.h(o',f') = v \\
	& \land v \in \textsf{OID} \\
	& \land  \textsf{ftype}(f) = \textsf{rcu} \\
 	&\implies\left\{ 
		\begin{array}{cl} 
			&  o=o' \land f=f' \\
			& \lor\textsf{unlinked} \in O(o) \\ 
			& \lor \textsf{unlinked} \in O(o') \\
			& \lor  (\_, o) \in \sigma.F \\ 
			& \lor  (\_, o') \in \sigma.F 
		\end{array}
		\right\}
\end{array}
\right\}
\]
\[
\textbf{A Reference Cross Read/Write Blocks-Iterator I}(\sigma,O,U,T)=
\left\{
\begin{array}{cl}
	& \forall tid,o \in \textsf{RCU} \ldotp \\ 
	& \sigma.s(x,tid) = o  \\
	& \land (x,tid) \notin U \\
&\implies \left\{
		\begin{array}{cl}
 			& \textsf{iterator } \, tid \in O(o) \\
 			& \lor \sigma.l=tid \land (\textsf{unlinked} \in O(o) \\
			& \lor (\_,o) \in \sigma.F) 
		\end{array}
	\right\}
\end{array}
\right\}
\]
\[
\textbf{Single Reader Reference-Iterator II}(\sigma,O,U,T) =
\left\{
\begin{array}{cl}
 & \forall tid \in \sigma.R, o,o' \ldotp \\
&\textsf{iterator } \, tid \in O(o)  \\
&\land \textsf{iterator} \, tid \in O(o') \\
&\implies \left\{ 
		\begin{array}{cl}
			o=o'
		\end{array}
	\right\}
\end{array}
\right\}
\]
\[
\textbf{References In Free List-Iterator III}(\sigma,O,U,T)=
\left\{
\begin{array}{cl}
	&\forall tid, o \ldotp \\ 
	& \textsf{iterator} \, tid \in O(o) \\
	& \land \exists T \ldotp (T,o) \in \sigma.F \\
& \implies \left\{ 
		\begin{array}{cl}
  			tid \in T
		\end{array}
	\right\}
\end{array}
\right\}
\]
%%%
\[
\textbf{Aliasing Path-Iterator IV} = 
\left\{
\begin{array}{cl}
\left\{
\begin{array}{cl}
& \forall x,y \ldotp \sigma.s(x,tid) = o' \land \sigma.s(y,tid) = o'' \\
&\land \textsf{iterator} \, tid \in O(o') \land \textsf{iterator} \, tid \in O(o'') \\
& \land P_1 = \textsf{PathToNode}(o = \textsf{UniqueRoot}, o') \\
&\land P_2 = \textsf{PathToNode}(o = \textsf{UniqueRoot}, o'') \\
& \land \textsf{IsSingleton}(P_1) \land \textsf{IsSingleton}(P_2) \\
&\textsf{IdenticalSequences}( P_1, P_2) \\
\end{array}
\right\}
& \implies \left\{
		\begin{array}{cl}
		 \sigma.h^{*}(o,P_1) = \sigma.h^{*}(o,P_2) 
		\end{array}
	\right\}
\end{array}
\right\}
\]
\[
\textbf{Disjoint Path-Iterator V} =
\left\{
\begin{array}{cl}
\left\{
\begin{array}{cl}
& \forall_{x,y \in VA} \ldotp \sigma.s(x,tid) = o' \land \sigma.s(y,tid) = o'' \\
&\land \textsf{iterator} \, tid \in O(o') \land \textsf{iterator} \, tid \in O(o'') \\
& \land P_1 = \textsf{PathToNode}(o = \textsf{UniqueRoot}, o') \\ 
&\land P_2 = \textsf{PathToNode}(o = \textsf{UniqueRoot}, o'') \\
& \land \textsf{IsSingleton}(P_1) \land \textsf{IsSingleton}(P_2) \\
&\lnot \textsf{IdenticalSequences}( P_1, P_2) \\
\end{array}
\right\}
& \implies \left\{
		\begin{array}{cl}
	             \sigma.h^{*}(o,P_1) \neq \sigma.h^{*}(o,P_2) 
		\end{array}
	\right\}
\end{array}
\right\}
\]
%%%
\[
\textbf{Unlinked Points To Unlinked-Unlinked I}(\sigma,O,U,T) =
\left\{
\begin{array}{cl}
&\forall o \ldotp \textsf{unlinked} \in O(o) \\
&\implies\left\{ 
	\begin{array}{cl}
		&\forall o',f '\ldotp \\ 
 		&\sigma.h(o',f') = o \\
		&\implies \left\{
			\begin{array}{cl} 
				&\textsf{unlinked} \in O(o') \\
				&\lor (\_,o') \in \sigma.F
			\end{array}
		\right\}
	\end{array}
	\right\}
\end{array}
\right\}
\]
\[
\textbf{Free List Inclusion-Unlinked II}(\sigma,O,U,T) =
\left\{
\begin{array}{cl}
	&\forall o \ldotp (T, o) \in \sigma.F \\
	&\implies \left\{
 	\begin{array}{cl}
		&\forall o',f' \ldotp \sigma.h(o',f') = o \\
		&\implies \left\{
   		\begin{array}{cl}
   	 		&(T',o') \in \sigma.F \\
    			&\land T' \subseteq T
    		\end{array}
 		\right\}
 	\end{array}
 	\right\}
\end{array}
\right\}
\]
\[
\textbf{Unlinked Not Iterator-Unlinked III}(\sigma,O,U,T) =
\left\{
\begin{array}{cl}
	&\forall o \ldotp \textsf{iterator} \, \sigma.l \in O(o) \\
	& \implies \left\{ 
	\begin{array}{cl}
		&\textsf{unlinked} \notin O(o)	
	\end{array}
	\right\}
\end{array}
\right\}
\]
\[
\textbf{Unlinked Refences In Free List-Unlinked IV}(\sigma,O,U,T) =
\left\{
\begin{array}{cl}
 \lnot (\forall o \ldotp \textsf{unlinked} \in O(o) \land (\_,o) \notin \sigma.F)
\end{array}
\right\}
\]
\[
\textbf{Fresh Points To Fresh-Fresh I}(\sigma,O,U,T) =
\left\{
\begin{array}{cl}
&\forall o \ldotp \textsf{fresh} \in O(o) \\
&\implies \left\{ 
          \begin{array}{cl}
	&\forall o',f' \ldotp  \sigma.h(o',f') = o \\
&\implies \left\{ 
	\begin{array}{cl}
	&\textsf{fresh} \in O(o')
	\end{array}
\right\}
\end{array}
\right \}
\end{array}
\right\}
\]
\[
\textbf{Fresh Reference In Write Block-Fresh II}(\sigma,O,U,T) =
\left\{
\begin{array}{cl}
& \forall tid,x,o \ldotp \sigma.s(x,tid) = o \\
&\land \textsf{fresh} \in O(o)\\
&\implies \left \{
	tid = \sigma.l
\right\}
\end{array}
\right\}
\]
\[
\textbf{Writer Not Reader}(\sigma,O,U,T) =
\left\{
\begin{array}{cl}
 \sigma.l \notin \sigma.R
\end{array}
\right\}
\]
\[
\textbf{Free List Threads Subset Of Readers-Freeing I}(\sigma,O,U,T) =
\left\{
\begin{array}{cl}
\forall o \ldotp (T,o) \in \sigma.F \implies T \subseteq \sigma.R
\end{array}
\right\}
\]
\[
\textbf{Type of Reference in Free List-Freeing II}(\sigma,O,U,T) =
\left\{
\begin{array}{cl}
&\forall tid,o \ldotp (T,o) \in \sigma.F \\ 
&\land \textsf{iterator} \, tid \in O(o)\\ 
&\implies \left\{ 
	tid \in T
\right\}
\end{array}
\right\}
\]
\[
\textbf{MustBeAllocated}(\sigma,O,U,T) =
\left\{
\begin{array}{cl}
 &\textbf{\textsf{MustBeAllocated}}_{0}(\sigma,O,U,T) = \{o | \mathsf{iterator} \, tid \in O(o)\} \\
 &\textbf{\textsf{MustBeAllocated}}_{n+1} (\sigma,O,U,T)= \{ o' | o \in \textbf{\textsf{MustBeAllocated}}_{n}(\sigma,O,U,T) \lor \sigma.h(o,f) =o'\}
\end{array}
\right\}
\]
\[
\textbf{InDomain}(\sigma,O,U,T) =
\left\{
\begin{array}{cl}
&\forall o,f',o' \ldotp  \sigma.h(o,f) = o' \\
&\implies \left\{ o' \in dom(\sigma.h) \right\}
	\end{array}
\right\}
\]

\caption{Definition of WellFormedness: Memory Invariants}
\label{fig:definitionWellFormed}
\end{figure*}

We view type environments denotationally, in terms of the set of instrumented states permitted by a given environment. Figure \ref{fig:denotingtypeenviroment} defines the type environment denotation :
\[ \llbracket \Gamma  \rrbracket_{tid} \] 
Type denotations are mostly straightforward, though they rely on a notion of partial separating conjunction ∗ of instrumented states. To define this separation, we must first define composition of instrumented states which is an operation $\bullet$ : $\mathcal{M} \longrightarrow \mathcal{M} \longrightarrow \mathcal{M}$ that is commutative and associative. It is defined over observation map as exclusive union of observed type states:

\[O_{1} \bullet O_{2}(loc) =  O_{1}(loc) \uplus O_{2}(loc)\]

Separation algebra is define as :

\begin{gather}
 * = ( \land,\bullet,\uplus,\uplus) 
\\
\textsf{iterator} \, tid \in O \rightarrow tid \in T
\\
\textsf{unlinked} \in O \rightarrow \sigma.l \in T
\\
\textsf{fresh} \in O \rightarrow \sigma.l \in T
\end{gather}

Partial separating conjunction then simply requires the existence of two states that compose:
\[ m \in P \ast Q   \stackrel{def}{=} \exists m' \ldotp \exists m'' \ldotp m' \in P \land m'' \in Q \land m \in m' \bullet m''  \]

[TODO : Explain previous separating conjuction clearly]

\subsection{Views Framework}
Our soundness proof builds upon a version of the Views Framework [15], which is in some sense a generalization of the ideas behind separation logic and rely-guarantee reasoning. We gave the definitions required by Views Framework, $\mathcal{M}_{RCU}$, $*$ and $\bullet$. We define some relations over $\mathcal{M}$ to enable Views Framework give us structure to the soundness proof as an embedding of the type derivations into the view's program logic.

Thread interference relation $\mathcal{R} \subseteq \mathcal{M} \times \mathcal{M}$  defines permissible interference on an instrumented state. The relation must distribute over composition:
\[
   \forall m_{1}, m_{2}, m\ldotp (m_{1} \bullet  m_{2}) \Longrightarrow \exists  m'_{1} m'_{2} \ldotp m_{1} \mathcal{R} m'_{1} \land m_{2} \mathcal{R} m'_{2} \land m \in m'_{1} \bullet m'_{2}
\]

where $\mathcal{R}$ is transitive-reflexive closure of $\mathcal{R}_{0}$ shown at Figure \ref{fig:threadInterferenceRel}. A unit $\in \mathcal{M}$, a state with all maps are empty, to $\bullet$ is closed with respect to $\mathcal{R}$.
\begin{figure*}[!htbp]
\[
(\sigma, O, U, T) \, \mathcal{R}_{0} \, (\sigma', O', U', T') = \left\{
 \begin{array}{ll}
  \sigma.l  \in  T \rightarrow \sigma.h = \sigma'.h \land\\
  \textsf{iterator} \, tid \in O(o) \rightarrow o \in dom(\sigma'.h) \land \\
  O = O' \land U = U' \land T = T' \land\\
 t \in T \rightarrow \forall x \ldotp \sigma.s(x,t) = \sigma'(x,t)
 \end{array}
 \right.
\]
\caption{Thread Interference Relation}
\label{fig:threadInterferenceRel}
\end{figure*}

In the Views Framework, program logic itself contains judgements of the form $\{ p \} C \{ q \}$ for views p and q and commands $C$. A denotation of static assertions are in terms of instrumented states : $\llbracket \Gamma \rrbracket_{tid}$. Proving derivation of typing, ($\llbracket  \Gamma \vdash C \dashv \Gamma \rrbracket$ ) forms the basis of soundness.

This rule shows how types changed soundly in a concurrent setting. Denotation of a typing context:
\[
 \llbracket \Gamma \rrbracket : \mathcal{P}(\textsf{LState})
\]

\begin{figure*}[!htbp]
\centering
\[
\begin{array}{l@{\;\;=\;\;}l}
\llbracket \, x : \textsf{rcuIterator}\,  \rrbracket_{tid}
&
\{ (s,h,l,R,F), O,U \mid \textsf{iterator } \, tid\in  O(s(x)) 
\land x \notin U
\}
\\

\llbracket \, x : \textsf{unlinked} \, \rrbracket_{tid}
&
\{ (s,h,l,R,F), O,U \mid \textsf{unlinked}\in  O(s(x)) 
\land l = tid 
\land x \notin U
\}

\\

\llbracket \, x : \textsf{rcuNext } f_1\,x_1, \ldots f_n\,x_n  \, \rrbracket_{tid}
&
\left
\{ 
\begin{array}{l|l}
(s,h,l,R,F), O,U 
&
\textsf{iterator } \, tid\in  O(s(x)) 
\land x \notin U\\
&{}\land  s(x_1) = h(s(x), f_1) \\
&{}\ldots \\
&{}\land s(x_n) = h(s(x), f_n)
\end{array}
\right\}
\\
%x : rcuNext ft : y
\llbracket \, x : \textsf{rcuNext} f_1 \ldots f_n \, | \, y  \, \rrbracket_{tid}
&
\left
\{
\begin{array}{l|l}
(s,h,l,R,F),O,U
&
  \textsf{iterator } \, tid \in  O(s(x)) \\
&{}\textsf{iterator } \, tid \in O(s(y)) \\
&{}\land x \notin U \land y \notin U \,  \\
&{}\land  s(x_1) = h(s(x), f_1)  \\ &{}\ldots  \\ &{}\land s(y) = h(s(x), f_n)
\end{array}
\right\}
\\
%x:rcuIterator | y
\llbracket \, x : \textsf{rcuIterator} \, | \text{Bounding Set} = \mathcal{P}(\textsf{VAR})  \, \rrbracket_{tid}
&
\left
\{
\begin{array}{l|l}
(s,h,l,R,F),O,U
&
  \forall y \in \text{Bounding Set} \ldotp \\
&{}\textsf{iterator } \, tid\in  O(s(y)) \\
&{}\textsf{iterator }\, tid \in O(s(x)) \\
&{}\land y \notin U \land x \notin U\\
&{}\land  s(y_1) = h(s(y), f_1) \\ &{} \ldots \\ &{}\land s(x) = h(s(y), f_n)
\end{array}
\right\}
\\
\llbracket \, x : \textsf{fresh } f_1\,x_1, \ldots f_n\,x_n \,  \rrbracket_{tid}
&
\left
\{ 
\begin{array}{l|l}
(s,h,l,R,F), O,U 
&
\textsf{fresh}\in  O(s(x)) 
\land x \notin U\\
&{}\land  s(x_1) = h(s(x), f_1) \\
&{}\ldots \\
&{}\land s(x_n) = h(s(x), f_n)
\end{array}
\right\}
\\


\llbracket \, x : \textsf{undef }\rrbracket_{tid}
&

\{ 
(s,h,l,R,F), O,U 
\mid
x \in U\}
\\

\llbracket \, x : \textsf{normal }\rrbracket_{tid}
&

\{ 
(s,h,l,R,F), O,U 
\mid
x \notin U\}
\\
\end{array}
\]
\caption{Denoting Types and Type Environments}
\label{fig:denotingtypeenviroment}
\end{figure*}

We then define this on typing contexts as 
\begin{gather*}
\llbracket x_1 : T_1, \ldots x_n : T_n \rrbracket_{tid,\textsf{RMO}} = \\
\llbracket x_1 : T_1 \rrbracket_{tid}*_{local} \ldots *_{local} \llbracket x_n : T_n \rrbracket_{tid} *_{local} \llbracket \textsf{RMO} \rrbracket_{tid}
\end{gather*}
where
$\llbracket \textsf{R} \rrbracket_{tid} = \{ (s,h,l,R,F), O,U  \mid tid \in R \}$, 
$\llbracket \textsf{M} \rrbracket_{tid} = \{ (s,h,l,R,F), O,U  \mid tid = l \}$, and
$\llbracket \textsf{O} \rrbracket_{tid} = \{ (s,h,l,R,F), O,U  \mid tid \neq l \land tid \notin R \}$, and


Given the specific definitions of composition and interference, the Views Framework defines a number of useful concepts to help structure and simplify the soundness proof. First, it defines a view as the subset of instrumented states that are stable under interference:
\[ \textsf{View} \stackrel{def}{=} \{ M \in \mathcal{M} | \mathcal{R}(M) \subseteq M\} \] such that: 
The program logic is proven sound with respect to views, and our denotation of type environments is a valid view (stable with respect to R). The framework also describes a useful concept called the view shift operator $\subseteq$, that describes a way to reinterpret a set of instrumented states as a new set of instrumented states. It requires that :
\[p \sqsubseteq q \stackrel{def}{\Leftrightarrow} \forall m\in\mathcal{M} \ldotp \lfloor p * \{ m \} \rfloor \subseteq \lfloor q* \mathcal{R}(\{m\})\rfloor\]
This specifies how the region information can be changed soundly. That is, we can only change the region information such that all other possible threads can maintain compatible views.

To use the Views Framework soundness result we need to show for each axiom, $\Gamma_{1} \vdash_{\textsf{RMO}} \alpha \dashv \Gamma_{2}$, that
\[
\forall r\ldotp   \llbracket \alpha \rrbracket  \lfloor \llbracket \Gamma_{1} \rrbracket_{\textsf{RMO},tid}  * \{m\} \rfloor \subseteq \\
  \lfloor \llbracket \Gamma_{2} \rrbracket_{\textsf{RMO},tid} * \mathcal{R}(\{m\}) \rfloor
\]
\begin{theorem}[Axiom Soundness]
For each axiom, $\Gamma_{1} \vdash_{\textsf{RMO}} \alpha \dashv \Gamma_{2}$, we must show
\[
\forall r\ldotp   \llbracket \alpha \rrbracket  \lfloor \llbracket \Gamma_{1} \rrbracket_{\textsf{RMO},tid}  * \{m\} \rfloor \subseteq  \lfloor \llbracket \Gamma_{2} \rrbracket_{\textsf{RMO},tid} * \mathcal{R}(\{m\}) \rfloor
\]
\end{theorem}

We can define the denotation typing judgement as: 

\[ \llbracket \Gamma_{1} \vdash C \dashv \Gamma_{2} \rrbracket \stackrel{def}{=} \{ \llbracket \Gamma_{1} \rrbracket_{tid}\} C \{ \llbracket \Gamma_{2} \rrbracket_{tid}\} \]

%%%
For soundness proof, we describe the important lemmas supporting soundness proof, and omit natural uninteresting lemmas. Lemmas of the proof are described in detail in Appendix \ref{sec:lemmas}.
%%%
[TODO: Matthew, Is it ok or enough to give pointer to the real proofs of these lemmas at appendix in here ?]
\begin{Lemma}[Safe ReadBlock Traversal]
\begin{gather*}
\llbracket x:=x.f \rrbracket \lfloor \llbracket \Gamma,x:\mathsf{rcuIterator} \rrbracket_{\textsf{R},tid} * \{m\}\rfloor\subseteq \\
\lfloor \llbracket \Gamma,x:\mathsf{rcuIterator} \rrbracket_{\textsf{R},tid}  * \mathcal{R}(\{m\})\rfloor 
\end{gather*}
\end{Lemma}
\begin{proof}
\end{proof}

[TODO: Explain what we prove and tell detailed is in Appendix]

\begin{Lemma}[Safe MutatorBlock Unlink]
\begin{gather*}
\llbracket x.f:=z \rrbracket = \lfloor \llbracket \Gamma,x:\mathsf{rcuNext} \, f \, y, y:\mathsf{rcuNext} \, f \, z \rrbracket_{M,tid} * \{m\}\rfloor  \subseteq 
\\ \lfloor \llbracket \Gamma,x:\mathsf{rcuNext} \, f \, z, y:\mathsf{unlinked} \rrbracket  * \mathcal{R}(\{m\})\rfloor 
\end{gather*}
\end{Lemma}
\begin{proof}
\end{proof}

[TODO: Explain what we prove and tell detailed is in Appendix]

\begin{Lemma}[Safe MutatorBlock Link]
\begin{gather*}
\llbracket x.f:=z \rrbracket \lfloor \llbracket \Gamma,x:\mathsf{rcuNext} \, f \, y \, , z:\mathsf{fresh} \, f \, y \rrbracket_{\textsf{M},tid} * \{m\}\rfloor \subseteq \\ \lfloor \llbracket \Gamma,x:\mathsf{rcuNext} \, f \, z \, , z:\mathsf{rcuNext} \, f \, y  \rrbracket_{\textsf{M},tid}  * \mathcal{R}(\{m\})\rfloor 
\end{gather*}
\end{Lemma}
\begin{proof}
\end{proof}

[TODO: Explain what we prove and tell detailed is in Appendix]

\begin{Lemma}[Safe MutatorBlock Allocate a Node]
\[
\llbracket y := new \rrbracket \lfloor \llbracket \Gamma,y:\_ \rrbracket_{\textsf{M},tid} * \{m\}\rfloor  \subseteq \lfloor \llbracket \Gamma,y:\mathsf{fresh} \rrbracket_{\textsf{M},tid}  * \mathcal{R}(\{m\})\rfloor 
\]
\end{Lemma}
\begin{proof}
\end{proof}

[TODO: Explain what we prove and tell detailed is in Appendix]

\begin{Lemma}[Safe MutatorBlock Free an Unlinked Node]
\begin{gather*}
\llbracket  \mathsf{delayedfree}(x) \rrbracket \lfloor \llbracket \Gamma,x:\mathsf{unlinked} \rrbracket_{\textsf{M},tid} * \{m\}\rfloor  \subseteq \\ 
\lfloor \llbracket \Gamma,x:\mathsf{undef} \rrbracket_{\textsf{M},tid} * \mathcal{R}(\{m\})\rfloor 
\end{gather*}
\end{Lemma}
\begin{proof}
\end{proof}

[TODO: Explain what we prove and tell detailed is in Appendix]
%%%

Using the soundness theorem from the Views framework ~\cite{views} with the previous theorem we can show our type system is sound.

\begin{theorem}[Type Soundness]
\[ \Gamma_{1} \vdash C \dashv \Gamma_{2} \Longrightarrow \llbracket \Gamma_{1} \vdash C \dashv \Gamma_{2} \rrbracket \]
\end{theorem}
\begin{proof}
By induction on the derivation $ \Gamma_{1} \vdash C \dashv \Gamma_{2}$. 
\end{proof}
